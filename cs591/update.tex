% Created 2019-03-26 Tue 21:33
% Intended LaTeX compiler: pdflatex
\documentclass[11pt]{article}
\usepackage[utf8]{inputenc}
\usepackage[T1]{fontenc}
\usepackage{graphicx}
\usepackage{grffile}
\usepackage{longtable}
\usepackage{wrapfig}
\usepackage{rotating}
\usepackage[normalem]{ulem}
\usepackage{amsmath}
\usepackage{textcomp}
\usepackage{amssymb}
\usepackage{capt-of}
\usepackage{hyperref}
\author{ethan}
\date{\today}
\title{}
\hypersetup{
 pdfauthor={ethan},
 pdftitle={},
 pdfkeywords={},
 pdfsubject={},
 pdfcreator={Emacs 25.1.1 (Org mode 9.1.14)}, 
 pdflang={English}}
\begin{document}

\tableofcontents

\section{Term Paper Update}
\label{sec:org52ba2ca}
Thus far, I have approximately followed my intended timeline. It is now approaching the end of March, and I have conducted sufficient research to feel prepared to begin working on my algorithm for text summarization with respect to a query. It appears that there are several algorithms which could fairly easily be adapted to this task. Many algorithms work by classifying sentences as summary sentences or non-summary sentences. To this end, I could expand the algorithm to also factor relevance into this classification. One problem that has come to my attention is that this is a somewhat novel problem, and as a result the existing measures and labeled datasets for evaluating text summarization algorithms are not applicable. I have already planned to provide some background discussion of text summarization methods. Since it will be difficult to provide analysis with regards to accepted metrics, I will instead expand this background discussion with my reasoning for selecting a particular text summarization method for modification along with some discussion of which other text summarization algorithms may be suitable for adaptation. Additionally, I will provide some subjective discussion of the results of this algorithm.

I have already found an open dataset to use as my corpus. I should be able to finish implementation of this algorithm while beginning my paper during the first couple weeks of the month. This would leave the remainder of the month for analyzing results and finishing up the paper.

My previously discussed publication outlets still seem relevant.
\end{document}